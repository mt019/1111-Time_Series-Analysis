\documentclass[UTF8,a4paper,14pt]{ctexart}
\usepackage[utf8]{inputenc}
\usepackage{amsmath}
\usepackage{amssymb}
\usepackage{amsfonts}
%for 字體
%https://tug.org/FontCatalogue/
% \usepackage[T1]{fontenc}
% \usepackage{tgbonum}
\usepackage[bitstream-charter]{mathdesign}
\usepackage[T1]{fontenc}
%\usepackage{bm}#粗體
%\usepackage{boondox-calo}
\usepackage{textcomp}
\usepackage{fancyhdr}%导入fancyhdr包
\usepackage{ctex}%导入ctex包
\usepackage{enumitem} %for在Latex使用條列式清單
\usepackage{varwidth}
\usepackage{soul} %for \ul
\usepackage{comment}%\begin{comment}\end{comment}
\usepackage{cancel}%\cancel{}
%\usepackage{unicode-math}
\usepackage{physics}% for derivative/partial derivative

\usepackage[dvipsnames, svgnames, x_11names]{xcolor}

\usepackage[low-sup]{subdepth}
\usepackage{subdepth}

\newcommand{\indep}{\perp \!\!\! \perp}

\usepackage{amsthm}

\usepackage{graphicx}
\graphicspath{ {./images/} }


% \DeclareMathOperator{\E}{\mathbb{E}}
\newcommand{\E}{{\rm I\kern-.3em E}}
\newcommand{\Var}{\mathrm{Var}}
\newcommand{\Cov}{\mathrm{Cov}}
\newcommand{\Corr}{\mathrm{Corr}}
% \DeclareMathOperator{\Var}{\textbf{Var}}
% \DeclareMathOperator{\Cov}{\textbf{Cov}}
% \DeclareMathOperator{\Corr}{\textbf{Corr}}


\DeclareMathSizes{10}{10}{7}{5}

\usepackage[a4paper, margin=1.5in]{geometry}

\usepackage{array, makecell} %


%中英文設定
%\usepackage{fontspec}
% \setmainfont{Computer Modern Sans Serif}
% \setmainfont{TeX Gyre Termes}
% \usepackage{xeCJK} %引用中文字的指令集
%\setCJKmainfont{PMingLiU}
% \setCJKmainfont{DFKai-SB}
% \setmainfont{Times New Roman}
% \setCJKmonofont{DFKai-SB}
\pagenumbering{arabic}%设置页码格式
\pagestyle{fancy}
\fancyhead{} % 初始化页眉
\fancyhead[C]{TSA\quad HW 04\quad  R10A21126\quad  WANG YIFAN\quad   \today}
%\fancyhead[LE]{\textsl{\rightmark}}
%\fancyfoot{} % 初始化页脚
%\fancyfoot[LO]{奇数页左页脚}
%\fancyfoot[LE]{偶数页左页脚}
%\fancyfoot[RO]{奇数页右页脚}
%\fancyfoot[RE]{偶数页右页脚}

% \title{{Econometrics HW 03}}
% \author{R10A21126}
% \date{\today}

%\fancyhf{}
\usepackage{lastpage}
\cfoot{Page \thepage \hspace{1pt} of\, \pageref{LastPage}}

\renewcommand{\headrulewidth}{0.1pt}%分隔线宽度4磅
%\renewcommand{\footrulewidth}{4pt}

\allowdisplaybreaks
\usepackage[english]{babel}
%\usepackage{amsthm}
\newtheorem{theorem}{Theorem}[section]
\newtheorem{corollary}{Corollary}[theorem]
\newtheorem{lemma}[theorem]{Lemma}


\usepackage[most]{tcolorbox}
% \newtcbtheorem{Problem}{\bfseries Problem}{enhanced,arc= 1 mm,boxrule=0pt,frame hidden,
%   colback = Pink!50!White,
%   coltitle=black,
%   top=0.15in,
%   attach boxed title to top left=
%   {xshift=1.5em,yshift=-\tcboxedtitleheight/2},
%   boxed title style={size=small,colback=LightPink!70!White}
% }{}
\definecolor{babyblue}{rgb}{0.54, 0.81, 0.94}

\newtcolorbox[auto counter]{mybox}[1]{
    enhanced,
    arc= 1 mm,boxrule=1pt,
    colframe=pink!90!white,
    colback=white,
    coltitle=black,
    % colback=blue!5!white,
    attach boxed title to top left=
    {xshift=1.5em,yshift=-\tcboxedtitleheight/2},
    boxed title style={size=small,
    % frame hidden,
    colback=LightPink!10!White},
    top=0.15in,
    % fonttitle=\bfseries,
    title= {#1},
    subtitle style={boxrule=0.4pt,
              colback=yellow!10!red!10!white},
    breakable
  }
\newtcolorbox[auto counter]{Problem}[1]{
    enhanced,drop shadow={blue!5!white},
    colframe=babyblue!50!,
    fonttitle=\bfseries,
    title=Problem ~\thetcbcounter. #1,
    %separator sign={.},
    coltitle=black,
    colback=blue!3,
    top=0.15in,
    breakable
  }

\newenvironment{solution}
  {\renewcommand\qedsymbol{$\blacksquare$}\begin{proof}[Solution]}
  {\end{proof}}

\theoremstyle{definition}
\newtheorem{definition}{Definition}[section]

%\theoremstyle{notation}
\newtheorem*{notation}{\underline{Notation}}
%\newtheorem*{convention}{\underline{Convention}}
\newtheorem*{convention}{\underline{Convention}}

\theoremstyle{remark}
\newtheorem*{remark}{Remark}

\newenvironment{amatrix}[2]{%% [2] for 2 parameters 
  \left[\begin{array}
    %{cc|cc}
    %  {@{}*{#2}{c}|c*{#1}{c}}
     {{}*{#1}{c}|c*{#2}{c}}
}{%
  \end{array}\right]
}
% For augmented matrix  
%https://tex.stackexchange.com/questions/2233/whats-the-best-way-make-an-augmented-coefficient-matrix



\usepackage{pgfplots}
\pgfplotsset{width=10cm,compat=1.18}
% \pgfplotsset{width=10cm,compat=0.9}
\usepackage{csvsimple}
\usepackage{enumitem}


\begin{document}

\begin{Problem}{}
  Given the model \(y_t=a+bt+c_t+x_t\), where \(a\), \(b\) are constants, \(c_t\)
  is deterministic and periodic with 
  period \(s\) and \(x_t\) is a SARIMA\((p,0,q)\times(P,1,Q)_s\). What is the model for \(w_t=y_t-y_{t-s}\)?
  
\end{Problem}

\begin{solution}\,\\
  For \(x_t\):
  \begin{equation}\
    \begin{aligned}
      \phi_p(B) \Phi_P(B^s) \nabla_s x_{t} = c + \theta_q(B)\Theta_Q(B^s)a_t
    \end{aligned}
  \end{equation}


  \begin{equation}\
    \begin{aligned}
      w_t &= y_t - y_{t-s}\\
      &= (a + bt + c_t + x_t)-(a + b(t-s) + c_{t-s} + x_{t-s})\\
      &= bs + c_t - c_{t-s} + x_{t} - x_{t-s}\\
      &= bs + x_{t} - x_{t-s}\\
      &= bs + \nabla_s x_{t} 
    \end{aligned}
  \end{equation}
  Therefore \({w_t}\) is ARMA\((p,q)\times(P,Q)_s\) with constant term of \(bs\).
\end{solution}

\begin{Problem}{}
  Identify the following as certain multiplicative SARIMA models:
  \begin{enumerate}[label=(\alph*)]
    \item \(y_t=0.5y_{t-1}+y_{t-4}-0.5y_{t-5}+a_t-0.3a_{t-1}\)
    \item \(y_t=y_{t-1}+y_{t-12}-y_{t-13}+a_t-0.5a_{t-1}-0.5a_{t-12}+0.25a_{t-13}\)
  \end{enumerate}
\end{Problem}

\begin{solution}\,\\
  (a)
  \begin{equation}\
    \begin{aligned}
      y_t - y_{t-4} &= 0.5(y_{t-1}-y_{t-5}) + a_t - 0.3a_{t-1}\\
      (y_t - y_{t-4}) - 0.5(y_{t-1}-y_{t-5}) &=  a_t - 0.3a_{t-1}\\
      (\nabla_4 y_t ) - 0.5(\nabla_4 y_{t-1}) &=  a_t - 0.3a_{t-1}\\
      (1-0.5B)\nabla_4 y_t &= (1-0.3)a_t
    \end{aligned}
  \end{equation}
  We can see that \(y_t\) can be modeled as ARIMA\((1,0,1)\times(0,1,0)_4\), with \(\phi_1 = 0.5\) and \(\theta_1 = 0.3\).\\
  \\
  (b)
  \begin{equation}\
    \begin{aligned}
      y_t &= y_{t-1}+y_{t-12}-y_{t-13}+a_t-0.5a_{t-1}-0.5a_{t-12}+0.25a_{t-13}\\
      \nabla y_t &= \nabla  y_{t-12} + a_t-0.5a_{t-1}-0.5a_{t-12}+0.25a_{t-13}\\
      \nabla y_t - \nabla  y_{t-12} &= a_t-0.5a_{t-1}-0.5a_{t-12}+0.25a_{t-13}\\
      \nabla\nabla_{12} y_t &=  (1-0.5B-0.5B^{12}+0.25B^{13})a_t\\
      \nabla\nabla_{12} y_t &=  (1-0.5B)(1-0.5B^{12})a_t\\
    \end{aligned}
  \end{equation}
  We can see that \(y_t\) is an ARIMA\((0,1,1)\times(0,1,1)_{12}\), with \(\Theta_1 = 0.5\), and \(\theta_1 = 0.5\).
\end{solution}


\begin{Problem}{}
  If the characteristic polynomial of an AR time series model is
  \[(1-1.6B+0.7B^2)(1-0.8B^{12})\]
\begin{enumerate}[label=(\alph*)]
  \item Is the model stationary?
  \item Identify the model as a certain SARIMA model.
  \end{enumerate}
\end{Problem}

\begin{solution}\,\\
  (a)\\
  We have the characteristic functions
  \begin{equation}\
    \begin{aligned}
      \Phi(B) &= (1-0.8B^{12})\\
      \phi(B) &= (1-1.6B+0.7B^2)
    \end{aligned}
  \end{equation}
  and
  \begin{equation}\
    \begin{aligned}
      \Phi &= 0.8\\
      \phi_1 &= 1.6\\
      \phi_2 &=-0.7
    \end{aligned}
  \end{equation}
We can see that
  \begin{equation}\
    \begin{aligned}
      \Phi_1 &= 0.8 &&< 1\\
      \phi_1 + \phi_2 &= 0.9 &&< 1\\
      \phi_2 - \phi_1 &=-2.3 &&< 1\\
      \left\lvert \phi_2\right\rvert  &= 0.7 &&<1
    \end{aligned}
  \end{equation}
  The roots of the characteristic function are greater than 1 in absolute value. Therefore, the model is stationary.\\
  \\
  (b)\\
  This is a SARIMA\((2,0,0)\times(1,0,0)_{12}\) model with the form
  \[\phi_{2}(B)\Phi_{1}(B^{12})y_t = a_t
  \]


\end{solution}
\pagebreak

\begin{Problem}{}
  Suppose \(y_t=y_{t-4}+a_t\) with 
  \[y_t=a_t \text{ for }t=1,2,3,4.\]
  \begin{enumerate}[label=(\alph*)]
    \item Find the variance function for \(y_t\).
    \item Find the autocorrelation function for \(y_t\).
    \item Identify the model for \(y_t\) as a certain SARIMA model.
    \end{enumerate}
\end{Problem}

\begin{solution}
  \begin{equation}\
    \begin{aligned}
      y_t &= y_{t-4} + a_t\\
      &= y_{t-8} + a_{t-4} + a_t\\
      &= y_{t-12} + a_{t-8} + a_{t-4} + a_t\\
      &\vdots
    \end{aligned}
  \end{equation}
  Let \(t=4k+i\) where \(i = 1,2,3,4\) and \(k = 0,1,2,3\ldots\)
  \begin{equation}\
    \begin{aligned}
      y_t &= \underset{k+1 \text{ items}}{\underbrace{a_i +a_{4+i} +a_{8+i} +\cdots +a_{4k+i}}}\\
    \end{aligned}
  \end{equation}
  (a)\\
  \begin{equation}\
    \begin{aligned}
      \Var[y_t] &= \Var[a_i +a_{4+i} +a_{8+i} +\cdots +a_{4k+i}]\\
      &=(k+1)\sigma_a^2
    \end{aligned}
  \end{equation}
  (b)\\
  Let \(s=4h+j\) where \(j = 1,2,3,4\) and \(h = 0,1,2,3\ldots\)
  \begin{equation}\
    \begin{aligned}
      \Cov[y_t, y_s] &= \Cov[\left\{a_i +a_{4+i} +a_{8+i} +\cdots +a_{4k+i}\right\},\left\{a_j +a_{4+j} +a_{8+j} +\cdots +a_{4h+j}\right\}]\\
      &=\begin{cases}
        0 & i\neq j\\
        (\min{\left\{{h,k}\right\}}+1)\sigma_a^2 & i=j
      \end{cases}
    \end{aligned}
  \end{equation}
  \begin{equation}\
    \begin{aligned}
  \Corr[y_t, y_s] &= \begin{cases}
    0 & i\neq j\\
    \frac{\min{\left\{{h,k}\right\}}+1}{\sqrt{(h+1)(k+1)}} & i=j
  \end{cases}
\end{aligned}
\end{equation}
(c)\\
This is a SARIMA\((0,0,0)\times(0,1,0)_4\) model with the form
\[\nabla_4^1 y_t = a_t\]


\end{solution}



\begin{Problem}{}

  Consider the famous time series data “co2” (monthly carbon dioxide through 11 years in Alert, Canada).
  \begin{enumerate}[label=(\alph*)]
    \item Fit a deterministic regression model in terms of months and time. Are the regression coefficients     significant? What is the adjusted R-squared? (Note that the month variable should be treated as categorical and transformed into 11 dummy variables.)
    \item Identify, estimate the SARIMA model for the co2 level.
    \item Compare the two models above, what do you observe?
    \end{enumerate}
\end{Problem}


\end{document}