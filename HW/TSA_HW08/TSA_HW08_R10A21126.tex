\documentclass[UTF8,a4paper,14pt]{ctexart}
\usepackage[utf8]{inputenc}
\usepackage{amsmath}
\usepackage{amssymb}
\usepackage{amsfonts}
%for 字體
%https://tug.org/FontCatalogue/
% \usepackage[T1]{fontenc}
% \usepackage{tgbonum}
\usepackage[bitstream-charter]{mathdesign}
\usepackage[T1]{fontenc}
%\usepackage{bm}#粗體
%\usepackage{boondox-calo}
\usepackage{textcomp}
\usepackage{fancyhdr}%导入fancyhdr包
\usepackage{ctex}%导入ctex包
\usepackage{enumitem} %for在Latex使用條列式清單
\usepackage{varwidth}
\usepackage{soul} %for \ul
\usepackage{comment}%\begin{comment}\end{comment}
\usepackage{cancel}%\cancel{}
%\usepackage{unicode-math}
\usepackage{physics}% for derivative/partial derivative

\usepackage[dvipsnames, svgnames, x_11names]{xcolor}

\usepackage[low-sup]{subdepth}
\usepackage{subdepth}

\newcommand{\indep}{\perp \!\!\! \perp}

\usepackage{amsthm}

\usepackage{graphicx}
\graphicspath{ {./images/} }


% \DeclareMathOperator{\E}{\mathbb{E}}
\newcommand{\E}{{\rm I\kern-.3em E}}
\newcommand{\Var}{\mathrm{Var}}
\newcommand{\Cov}{\mathrm{Cov}}
\newcommand{\Corr}{\mathrm{Corr}}
% \DeclareMathOperator{\Var}{\textbf{Var}}
% \DeclareMathOperator{\Cov}{\textbf{Cov}}
% \DeclareMathOperator{\Corr}{\textbf{Corr}}


\DeclareMathSizes{10}{10}{7}{5}

\usepackage[a4paper, margin=1.5in]{geometry}

\usepackage{array, makecell} %


%中英文設定
%\usepackage{fontspec}
% \setmainfont{Computer Modern Sans Serif}
% \setmainfont{TeX Gyre Termes}
% \usepackage{xeCJK} %引用中文字的指令集
%\setCJKmainfont{PMingLiU}
% \setCJKmainfont{DFKai-SB}
% \setmainfont{Times New Roman}
% \setCJKmonofont{DFKai-SB}
\pagenumbering{arabic}%设置页码格式
\pagestyle{fancy}
\fancyhead{} % 初始化页眉
\fancyhead[C]{TSA\quad HW 04\quad  R10A21126\quad  WANG YIFAN\quad   \today}
%\fancyhead[LE]{\textsl{\rightmark}}
%\fancyfoot{} % 初始化页脚
%\fancyfoot[LO]{奇数页左页脚}
%\fancyfoot[LE]{偶数页左页脚}
%\fancyfoot[RO]{奇数页右页脚}
%\fancyfoot[RE]{偶数页右页脚}

% \title{{Econometrics HW 03}}
% \author{R10A21126}
% \date{\today}

%\fancyhf{}
\usepackage{lastpage}
\cfoot{Page \thepage \hspace{1pt} of\, \pageref{LastPage}}

\renewcommand{\headrulewidth}{0.1pt}%分隔线宽度4磅
%\renewcommand{\footrulewidth}{4pt}

\allowdisplaybreaks
\usepackage[english]{babel}
%\usepackage{amsthm}
\newtheorem{theorem}{Theorem}[section]
\newtheorem{corollary}{Corollary}[theorem]
\newtheorem{lemma}[theorem]{Lemma}


\usepackage[most]{tcolorbox}
% \newtcbtheorem{Problem}{\bfseries Problem}{enhanced,arc= 1 mm,boxrule=0pt,frame hidden,
%   colback = Pink!50!White,
%   coltitle=black,
%   top=0.15in,
%   attach boxed title to top left=
%   {xshift=1.5em,yshift=-\tcboxedtitleheight/2},
%   boxed title style={size=small,colback=LightPink!70!White}
% }{}
\definecolor{babyblue}{rgb}{0.54, 0.81, 0.94}

\newtcolorbox[auto counter]{mybox}[1]{
    enhanced,
    arc= 1 mm,boxrule=1pt,
    colframe=pink!90!white,
    colback=white,
    coltitle=black,
    % colback=blue!5!white,
    attach boxed title to top left=
    {xshift=1.5em,yshift=-\tcboxedtitleheight/2},
    boxed title style={size=small,
    % frame hidden,
    colback=LightPink!10!White},
    top=0.15in,
    % fonttitle=\bfseries,
    title= {#1},
    subtitle style={boxrule=0.4pt,
              colback=yellow!10!red!10!white},
    breakable
  }
\newtcolorbox[auto counter]{Problem}[1]{
    enhanced,drop shadow={blue!5!white},
    colframe=babyblue!50!,
    fonttitle=\bfseries,
    title=Problem ~\thetcbcounter. #1,
    %separator sign={.},
    coltitle=black,
    colback=blue!3,
    top=0.15in,
    breakable
  }

\newenvironment{solution}
  {\renewcommand\qedsymbol{$\blacksquare$}\begin{proof}[Solution]}
  {\end{proof}}

\theoremstyle{definition}
\newtheorem{definition}{Definition}[section]

%\theoremstyle{notation}
\newtheorem*{notation}{\underline{Notation}}
%\newtheorem*{convention}{\underline{Convention}}
\newtheorem*{convention}{\underline{Convention}}

\theoremstyle{remark}
\newtheorem*{remark}{Remark}

\newenvironment{amatrix}[2]{%% [2] for 2 parameters 
  \left[\begin{array}
    %{cc|cc}
    %  {@{}*{#2}{c}|c*{#1}{c}}
     {{}*{#1}{c}|c*{#2}{c}}
}{%
  \end{array}\right]
}
% For augmented matrix  
%https://tex.stackexchange.com/questions/2233/whats-the-best-way-make-an-augmented-coefficient-matrix



\usepackage{pgfplots}
\pgfplotsset{width=10cm,compat=1.18}
% \pgfplotsset{width=10cm,compat=0.9}
\usepackage{csvsimple}
\usepackage{enumitem}


\begin{document}

\begin{Problem}{}
  Suppose the annual sales (in millions) of company A follow an AR(2) model:
  \begin{equation}\
    \begin{aligned}
      y_t= 5+1.1y_{t-1}-0.5 y_{t-2} +a_t
    \end{aligned}
  \end{equation}
where \(\sigma_a^2=2\).
\begin{enumerate}[label=(\alph*)]
  \item Show that the \(\psi_1\) in the random shock form is also 1.1. 
  \item If the sales for 2005, 2006, and 2007 were 9, 11, and 10, respectively, forecast the sales for 2008 and 2009.
  \item Calculate the 95\% confidence interval of the 2008 forecast in (b).
  \item If we now know the real sales of 2008 is 12, update your forecast for 2009.
\end{enumerate}

\end{Problem}

\begin{solution}\,\\
(a)\\
  \begin{equation}\
    \begin{aligned}
      y_t &= 5+1.1y_{t-1}-0.5 y_{t-2} + a_t\\
      &=5+1.1(5+1.1y_{t-2}-0.5 y_{t-3} + a_{t-1})-0.5 y_{t-2} + a_t\\
      &=5+5.5+1.21y_{t-2}-0.55y_{t-3}+1.1a_{t-1}-0.5 y_{t-2} + a_t\\
      &\vdots
    \end{aligned}
  \end{equation}  
  Put \(y_t\) in random shock form:
  \begin{equation}\
    \begin{aligned}
      y_t = a_t+ \psi_1 a_{t-1} +\psi_2 a_{t-2}+\cdots
    \end{aligned}
  \end{equation}
  If we then equate coefficients of \(a_t\), we get the recursive relationships
  \begin{equation}\
    \left.
    \begin{aligned}
      \psi_0 &=1\\
      \psi_1-\phi_1\psi_0&=0\\
      \psi_j-\phi_1\psi_{j-1}-\phi_2\psi_{j-2}&=0 \text{ for } j=2,3\ldots
    \end{aligned}
    \right\}
  \end{equation}  
  \begin{equation}
    \begin{aligned}
      \psi_1-(1.1)(1)&=0\\
      \psi_1&=1.1
    \end{aligned}
  \end{equation}
  (b)\\
  \begin{equation}
    \begin{aligned}
      \hat{y}_{2007}(1)=\hat{y}_{2008}&= 5+1.1 y_{2007}-0.5 y_{2006}\\
      &=5+1.1(10)-0.5(11)\\
      &=10.5
    \end{aligned}
  \end{equation}
  \begin{equation}
    \begin{aligned}
      \hat{y}_{2007}(2)=\hat{y}_{2009}&= 5+1.1 \hat{y}_{2008}-0.5 y_{2007}\\
      &=5+1.1(10.5)-0.5(10)\\
      &=11.55
    \end{aligned}
  \end{equation}
  (c)\\
  The 95\% confidence interval of \(\hat{y}_{2008}\) is
  \begin{equation}
    \begin{aligned}
      \left(\hat{y}_{2007+1}-z_{0.05/2}\sqrt{\Var[e_{2007}(1)]},\hat{y}_{2007+1}+z_{0.05/2}\sqrt{\Var[e_{2007}(1)]}\right)
    \end{aligned}
  \end{equation}
  Since we have \(z_{0.025}=1.96\), and
  \begin{equation}
    \begin{aligned}
      \Var[e_{2007}(1)]&=\sigma_a^2\left\{1\right\}=2
    \end{aligned}
  \end{equation}
  Therefore,
  \begin{equation}
    \begin{aligned}
      &=\hat{y}_{2007+1}\pm z_{0.05/2}\sqrt{\Var[e_{2007}(1)]}\\
      &=10.5\pm (1.96)(\sqrt{2})\\
      &=10.5\pm 2.77
    \end{aligned}
  \end{equation}
  Thus, the 95\% confidence interval of \(\hat{y}_{2008}\) is
  \begin{equation}
    \begin{aligned}
      \left(7.73, 13.27\right)
    \end{aligned}
  \end{equation}
  (d)\\
  Given that the sales in 2008 turn out to be \$12 million, update the forecast for 2009.
  With the updating rule:
  \begin{equation}
    \begin{aligned}
      \hat{y}_{T+1}(l)=\hat{y}_{T}(l+1)+\psi_1 a_{T+1}
    \end{aligned}
  \end{equation}
  where \(a_{T+1}\) is the new residual as the time rolls to \(T+1\), and can be estimated as \[\hat{a}_{T+1} = y_{T+1}-\hat{y}_{T}(1)\]
  Thus,
  \begin{equation}
    \begin{aligned}
      \hat{y}_{2007+1}(1)&=\hat{y}_{2007}(1+1)+\psi_1 a_{2007+1}\\
      &=11.55+1.1(12-10.5)\\
      &=13.2
    \end{aligned}
  \end{equation}


\end{solution}

\end{document}